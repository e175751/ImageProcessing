\documentclass[a4paper,11pt,titlepage]{bxjsarticle}
\usepackage{listings}
\usepackage[dvipdfmx]{graphicx}    
\usepackage{amsmath}
\usepackage{fancybox,ascmac}
\usepackage{url}
\title{画像処理 レポート1}
\author{175751C 宮城孝明}
\date{\today}
\begin{document}
\maketitle
\tableofcontents
\clearpage
\section{環境構築}
  1 開発環境構築の手順は、anaconda3-2019.03を使用し
  2 conda create -n 環境名 python=バージョン
  3 source activate (環境名)
  4 pip install opencv-python
  5 pip install Pillow
  6 pip install numpy
\section{コード(python opencv)}
\begin{lstlisting}[basicstyle=\ttfamily\footnotesize, frame=single]
import os
import cv2

def main():
    img = "./img_89.png"
    img = imread(image_url, cv2.IMREAD_GRAYSCALE)
    cv2.imshow('image', img)

if __name__ == "__main__":
    main()
\end{lstlisting}
\section{コード(python pil)}
\begin{lstlisting}[basicstyle=\ttfamily\footnotesize, frame=single]
import os
import PIL import Image
import numpy as np

def main():
    image_url = "./img_86.png"
    img = Image.open(image_url)
    img.show()

if __name__ == "__main":
    main()
\end{lstlisting}



\begin{thebibliography}{99}
\bibitem{book1}
\bibitem{book2}
\end{thebibliography}
\end{document}


